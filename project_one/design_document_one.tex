\documentclass[12pt]{article}
\usepackage{graphicx} % Required for inserting images
\usepackage[letterpaper, margin=1in]{geometry}
\usepackage{newtxtext,newtxmath} 
\usepackage{setspace}
\singlespacing
\usepackage{sectsty}
\sectionfont{\bfseries} 


\title{Project $1$ \\ Design Document}
\author{Team number: $n$ \\ Ben Logan, Maksym Makarchuk}
\date{\today}

\begin{document}

\maketitle

\section{System Requirements}
This system is designed to read and solve a Sudoku puzzle. 
The first requirement is that the system must accept a specified 
set of parameters as input. These parameters consist of the group 
ID, the algorithm type, the puzzle type, and the path to the puzzle 
input file. To satisfy this requirement, a dedicated Jupyter Notebook 
cell will be implemented to collect these variables at the start of 
execution, after which they will be stored as global values and used 
consistently throughout the system. This ensures a standardized interface 
for controlling solver behavior.

The second requirement is that the system must be capable of reading 
a Sudoku puzzle from an input file. This means that the program must 
correctly interpret the path parameter, open the file, and parse its 
contents into an internal representation. Each input file is a plain-text 
document containing a 9×9 grid, where “?” symbols represent unknown 
values and digits represent fixed ones. To achieve this, the system will 
include a parser function that validates the format and converts the 
file into a two-dimensional Python list, thereby providing a structured 
puzzle state for subsequent processing. This design guarantees both 
robustness to malformed input and compatibility across all provided files.

The third requirement is that the system must solve puzzles of varying 
difficulty, specifically easy, medium, hard, and extreme. This requirement 
implies that the solver must be general enough to handle complex constraint 
interactions and not be limited to baseline cases. To meet this condition, 
the system will be validated on all sixteen provided puzzles, with four 
examples drawn from each difficulty level. Successful performance across 
this spectrum will demonstrate that the solver is both flexible and robust.

The fourth requirement is that the system must support multiple solving 
algorithms. These include simple backtracking, backtracking with forward 
checking, backtracking with arc consistency, local search using simulated 
annealing with the minimum-conflict heuristic, and local search using a 
genetic algorithm with a penalty function and tournament selection. This 
requirement necessitates a modular solver architecture, in which each 
algorithm is implemented independently but accessed through a unified 
interface. The algorithm type specified by the user will determine which 
method is executed. Such a design not only satisfies the current project 
specifications but also enables future extension by incorporating additional 
algorithms without altering the solver’s core structure.

The final requirement is that the solved Sudoku puzzle must be saved as 
an output file in the same directory as the Jupyter Notebook. This 
requirement implies that the system must generate a text file solution 
in the same format as the input, thereby allowing for straightforward 
validation. To accomplish this, the program will include an output 
function that creates a new text file, names it according to the chosen 
parameters, and writes the completed 9×9 grid to it. This ensures that 
each solution is preserved in a consistent, reproducible manner.

Together, these requirements define a flexible and modular Sudoku 
solver capable of handling diverse puzzle types and algorithmic 
strategies while maintaining input–output consistency and reproducibility.
\section{System Architecture}
\section{System Flow}
\section{Test Strategy}
\section{Task Assignments and Schedule}

\end{document}